% This is file JFM2esam.tex
\documentclass{hmj}
\usepackage[UTF8]{ctex}         %导入 ctex 宏包,添加中文支持
\usepackage{graphicx}
\usepackage{amsmath}
\usepackage{epstopdf, epsfig}
\usepackage{subfigure}
\usepackage{color}
\usepackage{soul}
\usepackage{tcolorbox}
\graphicspath{{fig/}}
%setting the path of graphics

\newtheorem{lemma}{Lemma}
\newtheorem{corollary}{Corollary}
\definecolor{dkgreen}{rgb}{0,0.6,0}

%建立超链接实现跳转
\usepackage{hyperref}
\hypersetup{hypertex=true,
            colorlinks=true,
            linkcolor=blue,
            anchorcolor=blue,
            citecolor=blue}
            
\usepackage[T1]{fontenc}
\usepackage{xcolor}
\usepackage{lmodern}
\usepackage{listings}

\definecolor{mygreen}{rgb}{0,0.6,0}
\definecolor{mygray}{rgb}{0.5,0.5,0.5}
\definecolor{mymauve}{rgb}{0.58,0,0.82}

\lstset{language=[90]Fortran,
  numbers=left,
  basicstyle=\ttfamily,
  keywordstyle=\color{blue},
  commentstyle=\color{mygreen}\itshape,    % comment style
  numberstyle=\tiny\color{mygray}, % the style that is used
  stringstyle = \color{magenta},
  morecomment=[l]{!\ },                    % Comment only with space after !
  breaklines = true,
  columns = fixed,
  basewidth = 0.5em,
}

\setulcolor{red} %set underlining color 下划线 \ul{}
\setstcolor{green} %set overstriking color 删除线 \st{}
\sethlcolor{yellow} %set highlighting color 高亮 \hl{}

\newcommand\Pra{\mbox{\textrm{Pr}}} % Prandtl number, cf TeX's \Pr product
\newcommand\Ra{\mbox{\textrm{Ra}}} % Rayleigh number
\newcommand\Gs{\mbox{\textrm{Gs}}} % Gershuni number


\renewcommand\contentsname{CONTENTS}  %将目录中文改为目录英文
            
\title{NOTE BOOKS}

\begin{document}

\maketitle

\begin{tcolorbox}[colback=black!5!white,colframe=white]
\begin{center}
{\color{dkgreen}{ %
The description of this files 文件说明%
 }}
 \end{center}
 \end{tcolorbox}
 
 \tableofcontents   %添加目录
  \newpage
  
 \section{PAPER}
\subsection{VIBRATION}
\subsubsection{\cite{Wang2020d}}
When \ul{horizontal vibration} is applied to the convection cell, a strong shear is induced to the body of fluid near the conducting plates, which destabilizes thermal boundary layers, vigorously triggers the eruptions of thermal plumes, and leads to \hl{a heat-transport enhancement by up to $600\%$}.

\begin{tcolorbox}[colback=blue!5!white,colframe=blue!75!black,title=The governing equations]
In the coordinate system, a vibration corresponding to a sinusoidal displacement $\vec{s}(t)=Acos(\Omega t)\vec{n}$ applied in the cells and its acceleration is $\vec{a}(t)=A\Omega^2cos(\Omega t)\vec{n}$. The corresponding acceleration thus consists of a gravitational and a vibrational part(aka, $g\vec{k} + A\Omega^2cos(\Omega t)\vec{n}$, only considering horizontal direction $\vec{n}=\vec{i}$). Horizontal vibration of amplitude $A$ and angular frequency $\Omega=2\pi f$ is applied to the convection cell in the $x$ direction. The governing equations  are solved in the Oberbeck-Boussinesq approximation
\begin{equation}
  \nabla\bcdot\boldsymbol{u^*}=0 
\label{Eq1.1}
\end{equation}
\begin{equation}
\begin{aligned}
  \dfrac{\p \boldsymbol{u^*}}{\p t^*} +(\boldsymbol{u^*}\bcdot\nabla)\boldsymbol{u^*}=-\dfrac{1}{\rho}\nabla p^*+\nu\nabla^{2}\boldsymbol{u^*}+\alpha T[g\vec{k} + A\Omega^2cos(\Omega t)\vec{i} ] 
\label{Eq1.2}
\end{aligned}
\end{equation}
\begin{equation}
  \dfrac{\p T}{\p t^*}+(\boldsymbol{u^*}\bcdot\nabla)T=\kappa\nabla^{2}T
\label{Eq1.3}
\end{equation}
 Introducing $u = u^*/u_{ref},\theta =  T/\theta_{ref}$ and $p = p^*/p_{ref}$, which $x_{ref}=H, u_{ref}=\sqrt{\alpha g H \Delta}, t_{ref}=x_{ref}/u_{ref},\theta_{ref}=\Delta,p_{ref}=\rho u_{ref}^2,\Ra=\alpha g H^3 \Delta /(\kappa \nu),\Pra = \nu/\kappa$ and $\omega = \Omega \sqrt{H/(\alpha g \Delta)}$. Divided by $\alpha g \Delta$, the  dimensionless governing equations read
\begin{equation}
  \nabla\bcdot\boldsymbol{u}=0 \\
  \label{Eq1.4}
\end{equation}
\begin{equation}
  \dfrac{\p \boldsymbol{u}}{\p t}+(\boldsymbol{u}\bcdot\nabla)\boldsymbol{u}=-\nabla p+\dfrac{1}{\sqrt{\Ra/\Pra}}\nabla^{2}\boldsymbol{u}+\theta[\vec{k} + \dfrac{\Ra_{os}}{\Ra} cos(\omega t)\vec{i}]
  \label{Eq1.5}
\end{equation}
\begin{equation}
  \dfrac{\p \theta}{\p t}+(\boldsymbol{u}\bcdot\nabla)\theta=\dfrac{1}{\sqrt {\Ra\bcdot \Pra}}\nabla^{2}\theta
  \label{Eq1.6}
\end{equation}
Introducing the dimensionless vibration amplitude $\delta = \Ra_{os}/\Ra=0.1$, which the oscilatory Rayleigh number $\Ra_{os}=A\Omega^2 \alpha \Delta H^3/(\nu\kappa)$. 
\end{tcolorbox}

Simulations are conducted by using the open source spectral element code Nek5000. In the present study, the number of spectral elements ($Nx \times Ny\times Nz$) increases with both $\Ra$ and $\omega$ . For example, it is increased from $64 \times 20\times 72$ to $128\times 40\times 160$ as {\color{red}$\omega$ increases from 0 to 1700} at $\Ra$ = $10^8$. The order of the Legendre polynomials is chosen to be 7 on each spectral element. All statistics are calculated over an averaging time of more than 500 dimensionless time units after the system has reached the steady state.

温度摄动在Nek5000中的实现(只考虑横向,纵向的影响较小).
\begin{tcolorbox}[colback=black!5!white,colframe=white]
\begin{lstlisting}[frame=lrtb]
! Horizontal vibration
subroutine userf
! Note :this is an acceleration term, NOT a force
! ffx will subsequently be multiplied by rho(x,t)
ffx = \delta*cos(\omega*t)*temp
ffy = 0
ffz = temp
end
\end{lstlisting}
\end{tcolorbox}
 
 It is seen that all data points collapse nearly on top of each other for all Ra and for both 3D and 2D cases. This signals that the Nu enhancements for all cases studied exhibit universal properties and are governed by the same vibrational convective mechanism, shown in Fig. \ref{fig1}. \hl{When the frequency of vibration exceeds a critical value, the induced shear becomes large enough to vigorously destabilize the boundary layers. }

%FIGURE
\begin{figure}
  \centering
  \centering
  \includegraphics[width=0.8\textwidth ,trim=0 0 0 0,clip]{./paper/WZS20_SA.pdf}
  %trim= left down right up
    %\SetEnglishCaption
  \renewcommand{\figurename}{Figure}
  \caption{Ratio $Nu(\omega)/Nu(0)$ as a function of the normalized frequency $\omega /\omega^*$(The best fits of the crossover function $y = log [10^{1/4} + (\omega/\omega^*)^{a/4}]^4$ to the respective data, from which the value of $\omega^*$ is gained).}
  \renewcommand{\figurename}{图}
\label{fig1}
\end{figure}
 
 \subsubsection{\cite{Yang2020}}
 Our modulation method is complementary to hitherto used concepts of using additional body force or modifying the spatial structure of the system to enhance heat transport, for example, adding surface roughness, shaking the convection cell, including additional stabilizing forces through geometrical modification, rotation, inclination, or a second stabilizing scalar field, etc.
 In this Letter, we numerically study modulated RB convection within a wide range (more than four orders of magnitude) of modulation frequency at the bottom plate temperature and observe \hl{a significant ($\approx 25\%$) enhancement in heat transport}, seen in Fig. \ref{fig2}.
 
共振调频在Nek5000中的实现.
\begin{tcolorbox}[colback=black!5!white,colframe=white]
\begin{lstlisting}[frame=lrtb]
! modulated RB convection
subroutine userbc
! a sinusoidal modulation signal to the bottom temperature
if(z .gt. 0.5)temp = -0.5
if(z .lt. 0.5)temp = 0.5 + cos(\omega *t)
end
\end{lstlisting}
\end{tcolorbox}
 
 %FIGURE
\begin{figure}
  \centering
  \centering
  \includegraphics[width=0.8\textwidth ,trim=0 0 0 0,clip]{./paper/YCWVSL20.pdf}
  %trim= left down right up
    %\SetEnglishCaption
  \renewcommand{\figurename}{Figure}
  \caption{(a) Modulated frequency dependence of the Nusselt number $Nu( f)$, normalized by $Nu_0 = Nu( f = 0)$, for different Rayleigh numbers and fixed $\Pra = 4.3$. (b) $Nu( f)/Nu_0$ for different Prandtl numbers and fixed $Ra = 10^8$. (c) Global $Re( f)$ normalized by the $Re_0 = Re( f = 0)$ for different Rayleigh numbers and fixed $\Pra = 4.3$.}
  \renewcommand{\figurename}{图}
\label{fig2}
\end{figure}

 
\subsubsection{\cite{Bouarab2019}}

\subsubsection{\cite{Mialdun2008}}
\hl{(Vertical vibration)} Vibrational convection refers to the specific flows that appear when a fluid with a density gradient is subjected to external vibration. Vibrations can suppress or intensify gravitational convection depending on the mutual orientation of vibration axis and thermal (compositional) gradient. {\color{red} The ratio $\Gs/\Ra$ describes the relative importance of thermovibrational and gravitational convective mechanisms, which the Gershuni  number $\Gs=(A\alpha\Omega\Delta H)^2/(2\nu\kappa)=\Ra_{os}^2/(2\Ra \omega^2)$.}

%FIGURE
\begin{figure}
  \centering
  \centering
  \includegraphics[width=0.8\textwidth ,trim=0 100 0 100,clip]{./paper/MRMS_08_PRL_vibration.pdf}
  %trim= left down right up
    %\SetEnglishCaption
  \renewcommand{\figurename}{Figure}
  \caption{Flow structures and isotherms obtained by numerical simulations. (a) $\Gs =7\times10^3$, (b) $\Gs=70\times10^3$}
  \renewcommand{\figurename}{图}
\label{fig3}
\end{figure}
It was found that for Gershuni numbers below the critical value (for the present configuration, $\Gs_{cr} =15\times 10^3$), a four-vortex pattern [Fig.  \ref{fig3}(a)] is established at the steady state. Above the critical value, this pattern becomes metastable; the time of its existence decreases with increasing the Gershuni number. With time, the transition to the steady state with one big vortex and two small vortices occurs [Fig. \ref{fig3}(b)]. The transition between two structures of mean flow predicted by theory is demonstrated.\ul{ The flow is weak and does not change much with $\Gs$ number while $\Gs < \Gs_{cr}$. The influence of residual gravity is rather large here. As the Gershuni number increases, the thermal deviation $\theta'$ grows indicating the dominant role of vibration.}

\subsection{BOUNDARY LAYER}
能量平衡分析\\

1、平均流的动能平衡分析\\

2、脉动流的动能平衡分析\\
Next we study the variance temperature profile and consider a two-dimensional convective flow over an semi-infinite horizontal plate, following Wang $et$ $al.$\citep[]{Wang2016,Wang2018b}. Multiplying by $\theta'$ and taking a time average, one obtains
\begin{equation}
\underbrace{\eta_0\langle\boldsymbol{u}\rangle\cdot\nabla\Omega}_{mean\ convection} + \overbrace{2\langle \theta'v'\rangle\cdot\nabla\Theta}^{production} + \underbrace{\nabla\cdot\langle v'\theta'^2\rangle}_{turbulent\  convection}= \overbrace{\kappa\eta_0\nabla^2\Omega}^{diffusion} - \underbrace{2\epsilon _\theta}_{thermal\ dissipation}
  \label{Eq.4-1}
\end{equation}

数值模拟的各项能量通过上式方程得到, 各项能量的理论曲线由下面的温度脉动剖面方程得到,如图\ref{Eq.4-9}, 点表示数值模拟的各项能量,线为相应能量的理论曲线. we have the new equation of temperature variance profile for slip length $b/H >0$ is
\begin{equation}
\begin{aligned}[c]
  (1+d\xi^x)\dfrac{d^2\Omega(\xi)}{d\xi^2}+(\beta+xd)\xi^{x-1}\dfrac{d\Omega(\xi)}{d\xi}
    -\dfrac{[d\Omega(\xi)/d\xi]^2}{2\Omega(\xi)}\\
  +\dfrac{\Delta^2}{2\eta_0}\dfrac{m^x\xi^x}{(1+m^x\xi^x)^{2n}}
-2\alpha\Omega(\xi)=0
  \label{Eq.4-9}
  \end{aligned}
\end{equation}

%FIGURE
\begin{figure}
  \centering
  \centering
  \includegraphics[width=0.8\textwidth ,trim=0 0 0 0,clip]{./paper/WX18.pdf}
  %trim= left down right up
    %\SetEnglishCaption
  \renewcommand{\figurename}{Figure}
  \caption{Loss and Gain of energy}
  \renewcommand{\figurename}{图}
\label{fig3}
\end{figure}

\subsection{LARGE-SCALE CIRCULATION}

\subsection{MHD}

\section{TURBULENCE COURSE}

\section{SIMULATION CODE}
\subsection{Nek5000 code in fortran}

Here is how one can easily insert other programming langages (like Fortran) in LaTex.
\begin{tcolorbox}[colback=black!5!white,colframe=white]
\begin{lstlisting}[frame=lrtb]
! comment
subroutine mySubroutine()
    write (*,*) "Hello"
end subroutine
program main
    implicit none
    call mySubroutine()
end program main
\end{lstlisting}
 \end{tcolorbox}

\subsection{OpenFoam code in C}


\section{DEEP LEARNING}

\subsection{python}

%%EQUATION
%\begin{equation}
%******
%\label{Eq.1-1}
%\end{equation}
%公式Eqs. (\ref{Eq.1-1})


%%TABLE
%\begin{table}
%\begin{center}
%\begin{tabular}{c}
%%\toprule
%\medskip
%***
%\end{tabular}
%\caption{***}
%\medskip
%\label{label1}
% \end{center}
%\end{table}
%Label \ref{label1}


%FIGURE
%\begin{figure}
%  \centering
%  \centering
%  \includegraphics[width=1.0\textwidth ,trim=0 0 0 0,clip]{***}
%  %trim= left down right up
%  \caption{***}
%\label{fig1}
%\end{figure}
%Figure \ref{fig1}.

\section{Appendix}
\subsection{Appendix A}\label{sec:Appendix A}
见\autoref{sec:Appendix A}

%BIB FILE
\bibliographystyle{hmj}
% Note the spaces between the initials
\bibliography{notes}

\end{document}
